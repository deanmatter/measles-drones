A brief review of relevant literature is undertaken in this chapter. In \S \ref{sec:lit_matSim}, various methods of modelling epidemics are presented and discussed. This is followed by a discussion of literature about vehicle routing problems in \S \ref{sec:lit_VRP}, with a particular focus on literature about vehicle routing problems involving UAVs. A discussion of resource allocation problems in epidemic response is given in \S \ref{sec:lit_VaccHumRes}, and the chapter is concluded with an overview of facility location problems relevant for the problem of selecting a distribution centre from which vaccines can be delivered by UAV.

\section{Mathematical and simulation modelling of infectious diseases}
\label{sec:lit_matSim}
Modelling infectious diseases can prove indispensable in saving lives by helping decision makers to understand epidemics better. There are a number of modelling methods which have been developed for this purpose.

\subsection{Compartmental models}
\label{sec:lit_SEIR}
In compartmental models, the affected population is divided into distinct categories. These models can use either discrete or continuous time steps, describing disease progression using difference equations and differential equations, respectively \cite{selvam_2015}. For the SEIR measles epidemic model (an adaptation of the SIR model), appropriate population categories are: \textit{Susceptible} (S), \textit{Exposed} (E), \textit{Infectious} (I), and \textit{Removed} (R) \cite{chitnis_2017}. Categories for \textit{Dead} (D), and \textit{Vaccinated} (V) can be included as well. The population progresses between categories in the model as described in Figure \ref{fig:lit_SEIR}. Note the assumption that only individuals in the S, E and R classes can be vaccinated, and only individuals in the I class can die and move to the D class.

\begin{figure}[ht!]{\textwidth}
    \centering
    \resizebox{0.5\textwidth}{!}{\input{SEIRVD.pdf_tex}}
    \caption{Progression of population between categories in the SEIR model.}
    \label{fig:lit_SEIR}
\end{figure}

The basic continuous-time SEIR model is described by the following system of differential equations \cite{IDM_2019, chitnis_2017}:
\begin{align*}
    \frac{dS}{dt} &= -\frac{\beta SI}{N} \\
    \frac{dE}{dt} &= \frac{\beta SI}{N} - \sigma E \\
    \frac{dI}{dt} &= \sigma E - \gamma I - \mu I\\
    \frac{dR}{dt} &= \gamma I + \mu I.
\end{align*}

This basic SEIR model requires transition rates $\beta, \sigma, \gamma \text{ and } \mu$, for movement between categories. Let $N = S + E + I + R$ be the total population size. The population $N$ is assumed to be constant, for simplicity (so, there are no births in the model, and no deaths other than those caused by measles). 
The most commonly used descriptor of the rate of spread of epidemics is the basic reproduction number, $R_{0}$, defined as the ``expected number of secondary cases produced by a single (typical) infection in a completely susceptible population" \cite{jones_2007}. 
Where $\tau$ is the probability of a susceptible individual contracting the disease from an infectious individual, $\Bar{c}$ is the average number of contacts per infectious individual, and $d$ is the number of days of contagiousness, $R_{0} = \tau \cdot \Bar{c} \cdot d$ \cite{jones_2007}.
Therefore, the transmission rate (or, effective contact rate) of the disease is $\beta = \tau \cdot \Bar{c}$. Linking this to $R_{0}$, $$\beta = \frac{R_{0}}{d}.$$ 
The value of $R_{0}$ offers a good description of the contagiousness of an epidemic, and this is integrated into SEIR models through the $\beta$ parameter, through the above relationship.

\subsection{Population-based versus individual-based epidemic simulation}
In epidemiology, the progression of an epidemic can be simulated at a population level, or an individual level. The population level offers a broader view of a network of individuals and the interactions between population subgroups. Simulations at the individual level can provide more detailed insight into the transmission of the epidemic from one individual to another, and allow more accurate representation of heterogeneity in the population. For example, population-based simulation often requires the simplifying assumption of uniform transmission rates for entire populations \cite{bansal2007individual}. 
Bansal \textit{et al.} \cite{bansal2007individual} considered a ``network perspective to quantify the extent to which real populations depart from the homogeneous-mixing assumption,'' and found that compartmental epidemic models which assume uniform transmission and contact rates can be effectively used when the populations modelled are almost homogeneous. In the more general and common case of population heterogeneity, though, they found that network models, which ``explicitly model heterogeneity,'' are more accurate for simulating epidemic progression. 

The benefit of more accurate representation of `changing levels of infectiousness' was also discussed by Aral \textit{et al.} \cite{aral1996overview}, who note how the epidemic curve can be influenced by varying probabilities of transmission. They state that ``interventions focused on those most likely to transmit infection \dots should generally have relatively greater impact than interventions focused on the general population throughout an epidemic'' \cite{aral1996overview}. Therefore, simulating an epidemic at an individual level can more precisely describe the spread of an epidemic than at a population level, although it requires more specific data and certain simplifying assumptions. As discussed above, in the case of a relatively homogeneous population, or in the absence of detailed data on transmission rates, a population-based epidemic simulation suffices.

\subsection{Spatial dynamics}
As stated by Connell \textit{et al.} \cite{connell2009comparison}, the three main approaches to epidemiological modelling are mathematical models, network theory, and agent-based models. Compartmental models such as the SEIR model are typical mathematical methods in which epidemics are simulated, which ``[provide] rigorous results and [are] the simplest to implement'' \cite{connell2009comparison}. Compartmental models, as discussed in \S \ref{sec:lit_SEIR}, are often modelled using differential equations, with spatial dynamics accounted for in the equations. A potential drawback of these models, though, is that only simple scenarios and interventions can be tested analytically without significant adaptation \cite{connell2009comparison}. 

On the other hand, network models for epidemic representation ``provide broader models of epidemic outbreaks than simple equation-based models,'' with a greater capability for analysing scenarios and interventions \cite{connell2009comparison}. 
Finally, agent-based simulation allows for detailed and precise intervention analysis, and is becoming a more popular epidemiological model as computers are more commonly used. This accuracy allows the spread of epidemics over geographic areas to be modelled more accurately, with non-uniform contact and transmission rates, as well as more specific agent movement tracking. However, agent-based models are difficult to validate, and require many assumptions and detailed input data \cite{connell2009comparison}.

Guofo \textit{et al.} \cite{goufo2014fractional} use a `travel rate' between each pair of cities in a network as part of a fractional SEIR model's differential equations, to model the spread of the epidemic between those cities. This concept of a travel rate (i.e. a proportion of each subpopulation moving to each other subpopulation daily) allows effective modelling of the spread of an epidemic over a geographical area.

In the case of agent-based models, agents can be assigned certain attributes governing their movements between locations. These attributes, which can vary for each member of a population modelled, allow accurate representation of the movements of individuals. This, coupled with varying contact rates and transmission rates for each agent, can result in high accuracy when modelling the spatial spread of an epidemic.

\subsection{Measles modelling}
There are differing methods of simulating the spread of measles in a network. Kelker \cite{Kelker1973} presented a two-dimensional random walk model on a discretised grid of a network to simulate epidemic spread, successfully simulating a past measles epidemic. Liu \textit{et al.} \cite{liu2015role} successfully used agent-based simulation to model the spread of measles in California, finding that short intervention delay and high vaccination coverage is required to prevent large-scale measles epidemics. The approach of agent-based simulation provides low-level detail about epidemic spread and dynamics, but requires many assumptions and is unnecessary for the purposes of this project.

For the simulation of measles, compartmental models are most commonly used. Allen, Jones and Martin \cite{allen1990mathematical} implemented a SIR compartmental model to simulate a 1989 measles epidemic in Texas, stating the SIR model to be the ``standard model for the spread of measles through a population." Despite this, they found the SIR model to be ``not particularly well suited for [their] purposes" due to the lack of an \textit{Exposed} category, which is necessary for modelling measles. Subsequently, they developed a SLIR model (more commonly referred to as a SEIR model) with discrete time steps, to model a localised measles epidemic. They used this model to study the effectiveness of the interventions implemented in the 1989 outbreak.
McLean and Anderson \cite{mclean1988measles} also used an adapted SEIR model specifically for the case of modelling measles progression in developing countries, although not for the context of outbreak response. 
Longini and Ira \cite{longini1988mathematical} used a discrete-time SEIR model coupled with air transportation data to model the spread of influenza across a network of cities. A similar approach to theirs shall be followed in this project; modelling measles instead of influenza, and using assumed migration data to model epidemic spread between rural locations on a smaller scale.

Recall that Objective 4 of this project involves investigating resource allocation strategies in general, in the hope of discovering improvements in cost and lives saved. There is not much general previous work on this topic. Hethcote and Waltman \cite{hethcote1973optimal} used dynamic programming to find an optimal, lowest-cost solution to a deterministic SIR model, with the cost function assumed to include ``money, equipment, personnel, supplies, the disruption of health care elsewhere, etc." Unfortunately, though, the selection of an appropriate cost function is difficult to quantify without context-specific data, and their work does not yield much insight into resource allocation decisions.

An oft-cited value for $R_{0}$ for measles is $R_{0} = 15$ \cite{guerra2017basic}, making measles one of the most contagious illnesses in existence. After being exposed, individuals take 10 days to become infectious and move from the $E$ to the $I$ class, and the number of days for which an individual is infectious before recovery is generally 8 \cite{who_2019}. The measles death rate per case in developing countries is often 10\% \cite{moss2007measles}, significantly worse than in developed countries. All of these values can be used in an adapted discrete-time SEIR model similar to the continuous-time model described in \S \ref{sec:lit_SEIR}, to simulate the progression of a measles epidemic.

\section{Vehicle routing problem}
\label{sec:lit_VRP}
Generally, vehicle routing problems (VRPs) involve finding efficient routes for vehicles, to deliver or pick up items or people along these routes. The most common model of the VRP considers ``a fleet of identical vehicles [making] deliveries to customers from a central depot'' \cite{fisher1995vehicle}. The model allows for multiple vehicles with differing capacities, and specifies costs for the direct paths between each pair of demand nodes. The objective of the VRP model is to find a route of minimum cost for each vehicle, such that each demand node in the network is visited, and no vehicle capacities are exceeded \cite{fisher1995vehicle}. VRPs can be applied with fixed routing, or with variable routing. In the latter, routes can be generated on a daily basis (instead of being fixed) -- which is better for the case of finding routes during an ever-changing epidemic. 

Fisher describes the Clarke \& Wright method as the ``best known route building heuristic,'' presenting a range of adaptations of the method \cite{fisher1995vehicle}. Essentially, each customer is initially served by a separate route, and then routes are iteratively merged into combined routes according to the cost reduction resulting from combining each pair of routes. The Clarke \& Wright heuristic is beneficial for variable routing within a simulation, since it is significantly faster than other route generation methods. Fisher also presents mathematical programming methods for VRPs, which are able to account for vehicle capacities better than heuristics. These mathematical programming methods approximate VRPs with the generalised assignment problem, and the set partitioning problem, and find optimal solutions to these approximations quickly. Mathematical programming is thus a better method of solving VRPs than heuristics such as Clarke \& Wright, but is more complicated to implement than the relatively simpler heuristics.

For a practical implementation, Balcik \textit{et al.} \cite{balcik2008last} modelled a ``last mile distribution problem,'' to find delivery schedules for a set of vehicles, and to select allocations of inventory in the context of humanitarian relief and disaster response. They draw parallels between the problem considered and the inventory routing problem (IRP), stating that, ``the main decisions in the IRP are the customer delivery times, the number of items to be delivered at each visit, and the delivery routes.'' These three factors are also important in the case of vaccine deliveries in epidemic response. The inventory allocations found in their model were selected according to supply constraints, vehicle capacities and delivery time limits, and the primary objective was the minimisation of cost. Other than the fact that their model considers disaster relief instead of epidemic response, there are many similarities between the problems considered in their study and the problems faced by epidemic interventions. 

For vehicle route generation in the model created by Balcik \textit{et al.}, a travelling salesman problem (TSP) is solved for each vehicle, finding routes of minimum travelling time. Of the routes generated, the best routes are selected according to the demand met at locations along the route. This is a good method for finding routes, although finding optimal solutions for TSPs is computationally expensive, as they admit. This poses a significant drawback to TSPs being repeatedly solved for each day of a simulation. The possibility of using a faster method of route generation such as the Clarke \& Wright heuristic is appealing in order to reduce the time taken to run each simulation.

\subsection{Vehicle routing problems for UAVs}
\label{sec:lit_vrpUAV}
There is a growing body of literature about delivery networks incorporating UAVs. 
Murray and Chu considered the case of UAVs being launched from delivery vehicles, allowing more flexibility in the delivery network, as a variant on vehicle-only VRPs \cite{murray2015flying}. They named this problem the `flying sidekick TSP (FSTSP)'. Poikonen \textit{et al.} also discuss this type of delivery network, naming the problem a `vehicle routing problem with drones (VRPD)' \cite{poikonen2017vehicle}. In this delivery network, UAVs land at the same vehicle they launched from, allowing the vehicle to potentially make other deliveries while the UAV is in flight. This means that the UAV could land at a different location to the launching point. Using vehicles and UAVs in tandem allows a larger delivery range to be covered, and for impassable roads to be circumvented. Further, locations requiring larger quantities to be delivered could be delivered to by vehicle, and other locations by UAV. Murray and Chu formulated a mixed integer program (MIP) to model this scenario, and propose simple heuristics to solve the MIP, after noting that optimal solutions took hours to find. This proposed delivery method offers promise for vaccine deliveries, but has the limitation that vehicle capacity is reduced by the space occupied by the UAV, and that only quadcopter UAVs could be used, decreasing flight range considerably.

Murray and Chu also formulated a `parallel drone scheduling TSP (PDSTSP)', to model the alternative UAV delivery scenario; where vehicle deliveries are performed along a TSP route, and UAV deliveries are made independently, directly to demand nodes from the distribution centre \cite{murray2015flying}. Again, this was modelled using a MIP, with a heuristic solution procedure proposed. This scenario is more useful than the previous one for vaccine deliveries; vehicles can deliver larger quantities of vaccines to reachable locations, and UAVs can deliver in smaller quantities, to those locations which are further away or unreachable by road.

Scott and Scott modelled a different delivery network consisting of both vehicle deliveries and UAV deliveries \cite{scott2017drone}. In this proposed network, a single distribution centre supplies multiple `drone nests' with stock, via vehicle deliveries. In turn, UAV deliveries are made to demand nodes, from these drone nests. They used mathematical programming to model this network, solving for a delivery schedule which minimises total weighted delivery time. This type of network is also a possibility for deliveries in the context of epidemic response, although it has the downside that setting up multiple drone nests requires large capital outlays and potentially lengthy setup delays. An advantage of this type of network is that it would potentially be able to cover areas hundreds of kilometres large. Therefore, networks containing multiple drone nests are useful for epidemic interventions spanning across large geographical areas and multiple cities.

Since UAVs have small payloads, constructing formal vehicle routing problems to determine routes via multiple demand nodes is unnecessary. As in the above delivery network models discussed, vehicles can be used for traditional vehicle routing problems, but UAVs are best used for direct, return flights to demand nodes and back. Thus, a model for a combined network of UAVs and vehicles, such as the PDSTSP, is appropriate for the case of epidemic response.

\section{Vaccine and human resource allocation problem}
\label{sec:lit_VaccHumRes}
Two significant resource allocation problems in epidemic response are the allocation of teams to locations in the network, and the delivery of vaccine stock to teams. The former is constrained by the number of available vaccination teams, and the latter by the number of vaccines which can possibly be delivered on a daily basis. A general linear programming model for resource allocation was formulated by Feldstein \textit{et al.}, with resource availability constraints and various objective functions discussed \cite{feldstein_1973}. This form of linear program can be effectively used for finding optimal solutions to resource allocation problems. 

In a more practical application, Branas \textit{et al.} \cite{branas2000trauma} investigated an allocation problem where ``trauma centres and aeromedical depots'' were optimally placed within a network. Locations of severe injuries at points in the network were used to formulate a binary integer programming problem, with the objective of maximising coverage of severe injuries. Their investigation involves allocating limited resources to locations in a network in order to best respond to emergency situations, and is similar to this project in that regard. However, the allocation of resources in their investigation is according to existing data instead of an ongoing epidemic simulation, and the objective of maximising coverage differs from the objective of minimising deaths in an epidemic. Thus, their work doesn't entirely apply to this project's context, although binary integer programming poses potential for use in team assignments and vaccine stock allocations.

In order to optimise the distribution of different resources in a general humanitarian aid context, Vitoriano \textit{et al.} \cite{vitoriano2011multi} present a goal programming formulation of the problem, taking multiple objectives into account. These objectives include cost and response time, among others. The use of goal programming is beneficial as there are always multiple simultaneous criteria to be minimised in epidemic response, and it allows for resource constraints to be imposed. A potential drawback is that goal programming is difficult to integrate into a simulation, and that data on costs is often unavailable in epidemic response -- the primary criteria is the minimisation of infections and deaths, rather than reductions in costs. Therefore, linear programming can be used for resource allocation instead; assigning teams and vaccine stock to locations in order to minimise the number of infections and deaths, with the reduction of cost as a secondary objective.

\section{Facility location problem}
\label{sec:lit_facLoc}
In order to set up a vaccine delivery network, a distribution centre (DC) is required. Vaccines would be delivered to this DC in large quantities, and then stored in refrigerators before being delivered to vaccination sites. If the DC were to be the flights base for a UAV network as well, it would also require UAV charging stations, and equipment for launching, landing, and communications. Thus, the DC requires a stable source of electricity. In order to launch flights from the DC, it would need to be located in an area far enough away from airports. To allow as many locations to be serviced by UAVs as possible, it would also need to be placed within flight range of most locations. These three constraints on the location of the DC are most important. Under the assumption that urban areas in the network have electricity and rural areas do not, the DC needs to be placed in an area of the network labelled as urban.

There are a host of location-allocation methods which may be used to find an optimal location for the DC, given a network of possible locations.  Given a list of $n$ potential facility locations, the p-median Problem finds $p$ optimal locations for facilities, minimising total transportation cost \cite{beresnev_2018}. The capacitated facility location problem (CFLP) includes a fixed cost when opening each facility, and limits facility service capacity -- which are both factors in this problem \cite{beresnev_2018_CFLP}. If the DC placement was not limited to only being in discrete locations, the continuous-space, multi-facility Weber problem could be more appropriate \cite[p.270]{church2009business}. A simple yet effective approach in the context of UAV delivery could be to form a convex combination of the locations in the network, with the weightings of each point being the total demand for vaccines at that point. This way, with the network plotted on a cartesian plane, the DC is essentially at the weighted `centre' of the network. A shortcoming of all of these methods, though, is that none attempt to maximise the coverage of locations in the network. 

In a network of $n$ nodes, the k-centre problem selects $k$ nodes to be in the `centre'. These are selected such that the maximum distance from any node outside the centre to its nearest node within the centre is minimised \cite{mount_2017}. This problem represents this case of placing a DC within the UAV network better -- the most important criteria is that locations are within range of the DC. 
Under the assumption of having a single DC only, the k-centre problem would have $k$ = 1, which can be formulated more simply as the minimax location problem. In the minimax location problem, a single location is selected such that the maximum distance from the selected location to any other location is minimised (when compared to other location selections). This is the most applicable DC placement method for the context of this study.