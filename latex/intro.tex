In recent years, Unmanned Aerial Vehicles (UAVs, or drones) have been developed rapidly and have been tested for a wide range of applications. One such application is in the field of humanitarian aid and epidemic response -- UAVs are already in regular use for the routine delivery of blood and medication, and research is being performed to investigate whether it is worthwhile and cost-effective to use them in emergency scenarios as well. In rural and hard-to-reach areas, UAVs can be particularly useful for delivery since they are not constrained by poor road networks, and can make a life-saving impact in the case of medical deliveries. This project investigates the potential usage of UAVs for the delivery of measles vaccines in a rural context such as this, in comparison to the existing land-based cold chain. Resource allocation strategies are also considered, to gain insight into potential improvements on current outbreak response efforts.

\section{Background}
\label{measles-background}
Measles is an extremely contagious disease caused by the measles virus, transmitted from infected to susceptible individuals, primarily through the air. It generally takes 10 to 14 days after exposure for the initial symptoms of measles infection to appear, followed by a rash 2 to 4 days thereafter. Subjects are contagious from roughly day 10 to day 18, and during this period, up to 90\% of unimmunized people they come into contact with will contract the measles virus from them \cite{cdc_2018}. 
In developing countries, particularly those whose citizens suffer from malnutrition, AIDS and Vitamin A deficiency, the death rate for measles cases is often as high as 10\%. This is nearly 100 times higher than in developed countries. This case-fatality rate can even rise to 30\%, particularly in hard-to-reach areas where large portions of the population have weak immune systems and have not yet been vaccinated \citep{world2017measles}. It is therefore imperative to prevent or control measles epidemics, particularly in developing countries.

\subsection{Measles epidemic prevention}
Fortunately, measles epidemics are preventable by vaccination, with a vaccine released in 1963. Prior to that, there were approximately 30 million cases and 2 million deaths worldwide every year, particularly in children and infants; almost 95\% of 15-year-olds would have been infected by measles before the vaccine's introduction \citep{world2017measles}.
The currently available vaccines are relatively cheap, and effective. From 2000 to 2015, the number of measles cases reduced by 75\%, and deaths reduced by 79\%. Globally in 2015, there were 35 cases per million people, and a total of 134\,200 deaths - a marked improvement from pre-1963. Vaccination immunises nearly all susceptible individuals, and can also be administered to patients within 72 hours of exposure to the virus to mitigate symptoms and reduce the severity of the infection \citep{world2017measles}. Outbreak response can thus include both post-exposure and pre-exposure vaccinations.

Despite these reductions in measles cases globally, vaccination of less developed nations has proven challenging due to limited healthcare and road infrastructure. 
After 2017, 85\% of children globally had received a first vaccination dose, with just 67\% having also received the second dose \citep{worldhealthorganization_2018}. 
An effective population coverage is only attained when 95\% of children under 15 years old are immune \cite{msf_2017}, and hence the current global vaccination levels are insufficient.
In 2015, there were an estimated 20.8 million infants worldwide (out of 141 million total births \citep{owidfertilityrate}) that did not receive the first vaccination dose - of which over 20\% were born in Nigeria, Ethiopia and the Democratic Republic of Congo (DRC) \citep{world2017measles}. 
Since many unvaccinated people are concentrated in hard-to-reach areas of developing nations, improved methods of reaching these susceptible individuals is a priority.

\subsection{Vaccine distribution challenges}
A major contributing factor to the poor vaccination rate in these and other developing countries is the lack of a proper distribution network through which to deliver vaccines and other supplies to vaccination sites. In many African countries, the majority of the transportation network consists of unpaved dirt roads, or poorly maintained tarmac. There are very few tunnels, bridges or highways, and these dirt roads often become impassable mud in rainy seasons, especially after persistent rains - which are common in equatorial African countries in particular. Even if roads are passable, it can take hours or days to deliver vaccines by road - which is often too slow of a response in the case of an outbreak or medical emergency.
This is a major problem because, ``just a third of Africans live within two kilometres of an all-season road" \citep{harris_2019}. That means two thirds of Africans are essentially unreachable during rainy seasons, and were an outbreak of any serious infectious disease to occur, there is no straightforward way to vaccinate people and curb the spread of the disease using the current cold chain. 

%The vaccine distribution network in many developing countries is antiquated and unnecessarily layered. These networks usually consist of a national Distribution Centre (DC) supplying regional DCs, which supply provincial DCs, which then supply district DCs. Finally, the district DCs deliver to the PODs \citep{humphreys_2011}.
%Instead, due to modern communications technology, much fewer layers are required - PODs can even communicate supply requirements directly with a single national DC. For the case of epidemic response, a smaller network with a single DC supplying PODs directly can often be implemented.

Balancing the vaccine inventory kept with vaccination teams is also highly challenging. Even with the assumption that all roads are passable year-round, demand for vaccines is uncertain. Excess stock causes increased cold storage costs and wastage (since vaccines have an expiry date), and a shortage of vaccines means that not enough people can be vaccinated. Since the cost of a dose of the measles vaccine is \$2.85 \cite{unicef_2019}, the cost of wastage can be significant. There is also often not enough cold storage space for vaccines at vaccination sites, particularly in the case of epidemic responses \citep{humphreys_2011}. On top of this, when roads become impassable due to weather or other factors, a complex, expensive and life-changing logistical challenge emerges.

\subsection{UAV delivery}
In recent years, UAVs have emerged as a possible alternative to the land-based cold chain for medical deliveries, with other applications including farming, warfare, firefighting, entertainment, and humanitarian aid, among others.
Specifically within the field of healthcare, there have already been attempts at UAV deliveries for medication, blood and vaccines.
Matternet, an American company, has used fully autonomous quadcopter UAVs in conjunction with Doctors Without Borders and UNICEF, to deliver medication in Haiti, the Dominican Republic, Switzerland, and New Guinea. Matternet, as well as UPS, DHL Parcel and Flirtey, have also recently embarked upon forays into routine UAV deliveries of medical supplies in developed countries as well. DHL Parcel and Flirtey's UAVs have reduced certain routine delivery times from 30 minutes to 8 minutes, and 90 minutes to 3 minutes, respectively \cite{scott2017drone}.
Finally, a US startup called Zipline currently delivers blood and vaccines routinely in parts of Rwanda, Ghana and Tanzania. Zipline's UAVs are fixed-wing instead of quadcopters, allowing a top speed of over 144km/h rather than 40km/h. Despite a slightly smaller payload capacity, Zipline's UAVs can service a much larger range than quadcopters (160km rather than 10-20km), and are therefore the UAVs that are assumed to be used for the purposes of this project.

Unfortunately, the use of UAVs for medical deliveries is not without caveats. There is the serious risk of collisions with airplanes and, thus, appropriate airspace legislation is required before regular deliveries can take place. There are other concerns regarding mechanical reliability, payload dropoff methods, and theft. Specifically within the context of vaccine delivery, major challenges include controlling the temperature at which vaccines are transported at throughout the flight, as well as a limited delivery range and small payload. However, should some of these challenges be overcome, the use of UAVs for vaccine delivery appears to be highly promising.

\section{Problem description}
Planned, routine vaccination campaigns, particularly against contagious diseases such as measles, are a priority in many nations. However, more drastic action is required in the case of an epidemic. In response to a 2005 measles outbreak in the DRC, Doctors Without Borders (MSF) vaccinated 104\,549 children in two weeks, at 18 vaccination sites around the port town of Matadi \cite{msf_2006}. In such a rapid response, available response teams need to be promptly deployed in the network, and an efficient and reliable delivery network for vaccines needs to be established as quickly as possible. Currently, vaccines are delivered to vaccination sites by truck, motorbike, or even by foot - depending on the road quality and accessibility of the site. Sometimes rivers even need to be crossed in makeshift ferries \cite{msf_2017}. The World Health Organisation (WHO) mentions ``...optimal strategies for reaching hard-to-reach populations [for vaccination]" \cite{world2017measles} as an important aspect of measles intervention which Operations Research can assist with. To that end, this project will compare team allocation and delivery strategies, and evaluate the possibility of using UAVs to deliver vaccines in an outbreak response scenario.

More specifically, the research question under consideration is the following: In the case of a rural measles outbreak, are UAVs an efficient alternative vaccine delivery method (given capacity constraints), and which factors and resource allocation strategies contribute most to the success of an intervention?

\section{Scope of project}
In the scope of this project, the only disease considered is measles -- although this approach can easily be adapted to other diseases by changing parameters. This project only considers delivery networks using UAVs similar to those used by Zipline, since their UAVs can cover the furthest range. Since this range is large enough to include a city and surrounds, the delivery network is assumed to have only a single DC. The temperature at which the vaccines are transported and stored is not considered. Weather and seasonality do have an effect on the contagiousness of measles, but this effect is also not modelled in the scope of the project. The entire network of locations is assumed to be closed, with no migration into or out of the network. Finally, it is assumed that vaccination is the only intervention occurring in the epidemic, and that there is no impact of any other healthcare activities on the population.

\section{Project objectives}
The objectives that shall be followed to address the problem outlined are:

\textbf{Objective 1:} Perform a literature review, considering literature about epidemic modelling and simulation, to establish methods which can be implemented in this project. 

\textbf{Objective 2:} Develop a simulation model for the spread of measles across a network of interacting populations.

\textbf{Objective 3:} Develop a resource allocation model for the daily allocation of human and vaccine resources across the network.

\textbf{Objective 4:} Investigate potential resource allocation strategies, and the effect of factors such as setup delay on interventions.

\textbf{Objective 5:} Evaluate possible improvements to interventions, such as UAV delivery.

\section{Structure of this report}
Immediately following this introduction, a brief review of relevant literature is undertaken in Chapter 2. This is followed by a detailed discussion of the methodology used for the simulation, in Chapter 3. Chapter 4 follows, with a presentation and discussion of the results found using the aforementioned methodology. Finally, conclusions are drawn in Chapter 5, before a list of references and appendix.