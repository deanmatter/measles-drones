
\subsection{Insights into controlling Ebola in west africa}
\cite{buyuktahtakin2018new} use a model to ``...determine the optimal amount, timing and location of resources that are allocated for controlling and infectious disease outbreak while accounting for its spatial spread dynamics." The purpose of the model was to minimize the total number of infections and fatalities under a limited budget and time period. They consider \emph{geographically varying} rates for disease transmission, as well as the movement of individuals between geographic regions, and varying treatment rates due to limited clinic capacities. A mixed-integer programming model is proposed, which minimises the number of infections and deaths due to Ebola, while considering the spread of the epidemic and the logistics required to control it. The migration of people within the risk area of the epidemic is also taken into account, as this can affect its spread.  Various classifications of the population are used in the model; Susceptible, Infected, Treated, Recovered, Dead, and Buried. The final classification is significant due to West African burial practices, which involve contact with the corpse during the burial procedure. \\
There is a risk to oversimplifying the modelling of spread of epidemic outbreaks by ignoring the spread of disease over geography, and the differing rates of treatment capacity. \cite[p.~2]{buyuktahtakin2018new} \\
``...treatment considerably lessens the disease’s impact, compared to the no-treatment case, regardless of the starting date"
``the cumulative number of treated individuals in- creases proportional to the available budget” \\
``In all cases, we observe that treatment considerably lessens the disease’s impact, compared to the no-treatment case, regardless of the starting date. However, the earlier the starting date of the intervention, the fewer the infections and funerals.” \\
``Our results have a one-to-one correspondence with the basic reproduction number, R 0 , which is defined as the expected number of secondary cases of a typical single infected individual during his or her entire infectious period, in a population which is entirely susceptible. If R 0 < 1, dis- ease transmission is not self-sustaining and is unable to generate a major epidemic. On the other hand, an epidemic is likely to occur whenever R 0 > 1. If $R_{0}$ = 1 , each existing infection is causing one new infection, and the disease will stay alive without epidemic.” \\
``This result shows that for each month of delay in intervention, the required optimal amount of budget to control the epidemic nearly doubles.”\\
``Although a significantly larger treatment budget is allocated in delayed-intervention cases, total infections and deaths by the end of the planning horizon are considerably larger compared to early treatment efforts”\\
``Our analysis suggests that in order to stop the spread of Ebola and consequently reduce fatalities, policymakers must focus primarily on reducing transmission and contact rates aside from the rapid introduction of treatment efforts”\\
``All results from the case study consistently show that early control reduces infection and fatalities considerably more compared to delayed intervention. While a late implementation of control intervention requires intense response efforts and a significantly larger treatment budget, it still leads to a substantially greater cumulative number of infections and deaths compared to the case when intervention is early.”\\

\subsection{Commonly used OR models for epidemiological modelling}
Methods of OR used for epidemiological modelling in the past, described by \cite{buyuktahtakin2018new}, include:
\begin{itemize}
    \item Compartmental model-based simulations - which can model disease dynamics fairly accurately, but can only consider a small number of intervention strategies. Further, simulations for optimization require a lot of computational power and time and use heuristics - which are not guaranteed to provide an optimal solution.
    \item Differential equation based models
    \item Resource allocation methods to minimise cost
    \item Network models
    \item Mathematical programming - including linear and integer programming, non-linear optimisation and stochastic programming. These methods can be useful because they analyse all intervention strategies and find the best possible one. These can be used in conjunction with simulation - epidemic spread and growth is simulated before being used as input in the model. These models are often used to find the best vaccination strategies.
    \item Facility location models - including maximal covering, optimal location-allocation and P-median allocation models
    \item Agent-based simulation - along with social networks, to model the interactions between people and transmission of the epidemic
\end{itemize}

\subsection{Wanying Article - on Anthrax attacks using Markov processes}
The model involves Markov processes and takes into account the progression of the anthrax disease in the medical and logistics response. \\
`` In particular, the approach allows us to capture dynamically the impact of different medical responses on the infected population. Dynamics are important because the time elapsed since the patient is infected till the moment he/she receives medical treatment has a major impact on the recovery rate, the survival rate and, therefore, on the number of deaths.” \citep{wanying2016modeling} \\
\textsc{Need to find disease stages for measles and work out at which point someone is infected and contagious, and use that in the model – cause drones need to get there quickly. Also, consider recovery or death rate at each point. } \\
``In other words, the main reason of the high death rate is that patients cannot get the medical help in time because of the limited logistics capacity and the short disease period.”\citep{wanying2016modeling} \\
``To sum up, it appears to us that, in all the cases, a large scale attack would require a very high dispensing antibiotics capacities, so the most effective way to reduce the number of causalities is to speed up the detection thus limiting the number of infected people."\citep{wanying2016modeling} \\

\subsection{Dasaklis - smallpox attack (disease progression model and LP for vaccine allocation)}
\citep{dasaklis2017emergency}
Transmission model is used for disease spread and dynamics, LP model for optimal distribution of vaccines to certain locations and populations. Some sensitivity analysis upon assumptions. \\
For smallpox, many people are susceptible since the disease is eradicated and routine vaccinations not needed. To what extent is this the same for measles? Also, vaccinations are risky – chance of illness and even death is considerable. Recommended strategy for vaccination in case of outbreak is targeted or ring vaccination – isolate and vaccinate close contacts. More and more people vaccinated if a high R0 basic reproduction number. \\
``From a logistical point of view,
the implementation of a broader vaccination campaign
should rely on the establishment of an emergency supply
chain and a series of decisions should be made regarding
the location, number and capacity of both the stockpile
centres and final Points of Dispensing (PODs), the
inventory level of medical supplies and commodities
held within these facilities as well as the replenishment
policies adopted, the assignment of these facilities to
serve certain sub-populations and, finally, the selection
of modes of distribution and relevant capacities.” \\
``In particular, the dynamics of the
spread of a smallpox outbreak and its interactions with
resource allocation decisions are considered. A modelling
approach is presented consisting of two modules. The
first module relates to disease’s progression, whereas the
second one relates to optimally distributing a set of supplies
(medical and ancillary) to affected sub-populations.” \\
``Although ring vaccination is highly recommended for
controlling a smallpox outbreak, it is questionable
whether such an intervention would yield optimal results
in the case of a large-scale bioterrorist attack” \\
``As the outbreak lasts several
weeks, birth/death processes have been excluded from
the disease progression model (they normally affect the
dynamics of endemic diseases over several years).” \\


\subsection{WHO Position Paper}
\citep{world2017measles}
``Increasing population immunity through vaccination is the most effective way to prevent outbreaks."

``There is no specific treatment for measles."

``In unimmunized or insufficiently immunized individuals, measles vaccine may be administered within 72 hours of exposure to measles virus to protect against disease." This reduces severity of symptoms and length of onset.
The standard vaccine dose is 0.5m$\ell$.
The vaccine needs to be stored below 8\textdegree C.
After a second vaccine dose, 95\% of children who did not become immune after their first vaccine dose become immune. But ``evidence indicates that a single dose
of correctly administered measles vaccine which results in seroconversion will afford lifelong protection for most healthy individuals."

Serious adverse affects to measles vaccination occur more frequently for multidose vials.

``Measles immunization saves high costs for the individuals
affected, their families and the national healthcare
system." Further, the measles vaccine is inexpensive.\\

``Programmatic
questions that should be addressed include: which
populations should be targeted for special immunization
efforts; optimal strategies for reaching hard-to-reach
populations, and adolescents and adults; how to
strengthen and enhance disease surveillance and
reporting; the best approaches to measuring vaccination
coverage; strategies to communicate the benefits of
measles vaccination and minimize vaccine hesitancy;
and the economic impact of the disease."\\
