This report introduced the problems faced by current outbreak response interventions, and the possibility of using UAVs in rural areas for humanitarian aid deliveries. The properties of measles and the method in which it is often simulated -- the compartmental model -- are also introduced. A short review of relevant literature and similar work is performed in Chapter 2, introducing methods that are used for the simulation. The methodology used to construct the simulation model in Python is presented in Chapter 3, which is followed by a detailed discussion on results obtained in Chapter 4. The results found in Chapter 4 provide some useful insight into improving epidemic response, with particular focus on the case of measles outbreaks in rural areas. The project was able to achieve all five objectives outlaid in the introduction; a review of relevant literature was conducted, and a simulation of a measles epidemic in a network was developed. The expected prevented exposures model was formulated and implemented for resource allocation decisions, and heuristic strategies for resource allocation were discussed and investigated. Finally, UAVs were evaluated as a vaccine delivery method, and certain other improvements to interventions were found.

In the sensitivity analysis performed on uncontrollable parameters, the model was found to be most sensitive to changes in $R_{0}$, and durations of exposure and infection. Among the controllable improvements found are aspects regarding teams, delivery methods, and the timing of the intervention. As the number of teams in the network increases, the number of deaths decreases, but this decrease reduces as even more teams are added. Since there is a tradeoff between cost and assigning more teams, 12-18 teams are recommended for a network such as the one investigated. However, this only applies to the case study considered; better methods for determining a good number of teams in general, pose potential for future work. Teams' working hours per day also play an important role; it was found that, on average in the whole vaccination network, for each minute above 8 hours worked daily, there are 1.46 deaths prevented. Further, having teams working in overlapping shifts rather than having a shared day off was shown to reduce deaths significantly. 

In repeated testing, simulations showed that the team allocation strategy employed has a much larger impact on the intervention than the choice of where to deliver vaccines to once teams are already placed. For the network tested, the strategy of sending teams to locations proportionally to those locations' populations was found to be the most effective team allocation method for reducing deaths, without considering cost. However, the best performing strategy when both deaths and cost are considered, was found to be the optimal expected prevented exposures strategy developed using linear programming. This strategy, compared to the population strategy, reduced average cost by \$81\,016 (a 22.4\% reduction), while only increasing the average number of deaths by 0.9\%, an almost negligible amount (although this may be significant, depending on the decision-maker). This EPE strategy, and the use of simulation during outbreak response to support decision-making, provides invaluable potential for improvement in interventions.

With regard to vaccine delivery methods, UAVs were shown to be able to meet demand for vaccines entirely, and are on par with vehicle delivery networks when 3 or more UAVs are used. Under the assumption that each UAV flight costs \$17, UAV delivery is cheaper than land-based delivery in areas with poor road infrastructure, and also reduces the average number of deaths. The percentage of roads impassable for which it is better to use UAVs than vehicles was estimated to be 12.5\%, based on the network used as a case study -- i.e. if any more than 12.5\% of roads in a network are impassable, UAVs would be cheaper and more effective than vehicles for vaccine delivery in the case of an outbreak. This percentage, however, only applies to the case study considered, and further research into a general percentage based on a range of different network structures is required.

UAVs with an average speed below 80 km/h or a capacity of below 30 vaccines were found to increase the average number of deaths, while all UAVs above these levels had almost identical performance. Therefore, these minimum standards need to be met for a UAV to suffice for a vaccine delivery network such as the one modelled. 
For the case study considered, the combination of one primary vehicle and one secondary, smaller vehicle were found to be sufficient for delivering vaccines on time. Further, results indicate that the measles vaccine would only need to remain potent for one day outside of the cold chain to maintain similar intervention success -- there is no improvement in the number of deaths if the vaccine lasts longer, although it would simplify deliveries and slightly reduce delivery cost.

Aside from team allocation, the most important aspects of epidemic response were found to be early detection, and swift, focused response. For each day longer than MSF's recommended 8 day response delay (the time between epidemic detection and the beginning of vaccinations), an average of 143 deaths was caused. It is thus imperative to reduce the time it takes to set up an intervention. Further, reducing migration by discouraging movement between locations in the network was shown to have a slight impact on the spread of the epidemic, albeit not as significant as better team allocations and earlier intervention. Finally, using targeted vaccination instead of untargeted, non-selective vaccination, was shown to almost halve the number of deaths and reduce cost by 39.2\%.

Other possible opportunities for future work include more realistic modelling of vehicle deliveries, with more intelligently selected routes, as well as better modelling of migration; perhaps using the radiation model instead of the gravity model. Agent-based simulation could be used to gain a different and more low-level insight into resource allocation strategies, and more team allocation strategies could be evaluated. If more comprehensive data can be gathered about another historical measles outbreak, the model can be validated on a different case study, and historical interventions can be compared with those tested and formulated in this report. Alternatively, a large number of networks could be generated programmatically for validation (and better determination of the percentage of closed roads for which it is beneficial to use UAVs instead of vehicles), allowing the effect of different network structures on the resource allocation models to be seen. Furthermore, global sensitivity analysis methods, such as partial rank correlation coefficients, could be implemented.

To reiterate the conclusions drawn and answer the research question posed in the introduction; UAVs were able to successfully meet vaccine demand in simulations, and reduced costs and deaths in cases where road quality was poor. Furthermore, the use of the EPE team allocation strategy with targeted vaccination, and having shorter intervention delays, were found to be the most significant factors in improving intervention success.